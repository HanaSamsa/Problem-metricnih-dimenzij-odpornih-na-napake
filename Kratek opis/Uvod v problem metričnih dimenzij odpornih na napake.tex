\documentclass[a4paper,10pt, fleqn]{article}
\usepackage[slovene]{babel}
\usepackage[utf8]{inputenc}
\usepackage[T1]{fontenc}
\usepackage{lmodern}
\usepackage{amsmath,amsfonts}
\usepackage{enumitem}
\usepackage{footmisc}
\pagestyle{empty}

\begin{document}

\section{Metrična dimenzija}

Množica $S \in V$ je v grafu $G$ razrešljiva, če za vsak par $x, y \in V(G)$ 
ostaja $s \in S,$ da velja $d(x, s) \neq d(y, s).$ Rečemo, da sta $x$ in $y$
razrešeni z vozliščem $s$. Množica $S$ je odporna na napake, če je 
$S \setminus \{v\}$ prav tako razrešljiva za vsak $v \in S.$ 

Metrična dimenzija neusmerjenega in povezanega grafa $G = (V, E)$ je najmanjša 
podmnožica nabora vozlišč $S \subset V$ z lastnostjo, da so vsa vozlišča v $V$ 
enolično določena z njihovimi razdaljami do vozlišč podmnožice $S$.

Primer uporabe metrične dimenzije je problem robotske navigacije. Pri tem pusitmo
robota, da navigira v nekem prostoru, ki je določen z grafom $G$. Pri tem so 
povezave grafa $G$ poti. Robot pošlje signal do posameznega niza vozlišč imenovanih 
orientacijske točke, da ugotovi kako daleč od njih se nahaja. Pri tem je določanje 
najmanjše množice orientacijskih točk in njihov položaj, da robot lahko enolično 
določi, kje se nahaja, simetričen problemu metrične dolžine. Problem nastane, 
če ena od teh točk ne deluje pravilno, kar pomeni da robot nima dovolj informacij 
za enolično določanje svoje lokacije. Pri teh težavah nam prav pridejo metrične 
dolžine, odporne na napake. 
Nabor za razreševanje, odporen na napake zagotavlja, da tudi če ena od imenovanih 
točk ne deluje pravilnno bomo dobili prave informacije.

Metrična dimenzija $G$, odporna na napake, je velikost najmanjše razčlenujoče 
množice $S,$ odporne na napake (v primeru robotske navigacije je to nabor za 
razreševanje) in jo označimo z $ftdim(G).$


\section{Matematična formulacija}

Imamo povezan in neusmerjen graf $G = (V, E),$ kjer je $V = \{1, 2, \ldots, n\}$
množica vozlišč in $\mid E \mid = m.$ Naj bo $d(u, v)$ najkrajša pot med vozliščema
$u, v \in V.$

\subsection{Celoštevilski linearni program}

V naslednjem celoštevilskem lineearnem programu naj velja $1 \leq u < v \leq n$ in 
$1 \leq i < j \leq n.$ Najprej definirajmo matriko koeficientov A. Spremenljivka $x_i,$ 
definirana s predpisom \eqref{eq:2} nam pove ali vozlišče $i$ pripada množici $S.$ 

\begin{equation}
    A_{(u, v), i} = \begin{cases}
        1, d(u, i) \neq d(v, i) \\
        0, d(u, i) = d(v, i)
    \end{cases}
\label{eq:1}
\end{equation}

\begin{equation}
    x_i = \begin{cases}
        1, i \in S \\
        0, i \notin S
    \end{cases}
\label{eq:2}
\end{equation} 

\begin{equation}
    \min \sum_{i = 1}^{n} x_i  
\label{eq:3}
\end{equation}

\begin{equation}
    \sum_{i = 1}^{n} A_{(u, v), i} \cdot x_i \geq 2, \text{ } 1 \leq u < v \leq n 
\label{eq:4}
\end{equation}

\begin{equation}
    x_i \in \{0, 1\}, \text{ } 1 \leq i \leq n
\label{eq:5}
\end{equation}

Enačba \eqref{eq:3} predstavlja najmanjšo podmnožico $S.$ Enačba \eqref{eq:4} pa nam da pogoj, da
je poodmnožica $S$ razrešljiva množica, odporna na napake. To pomeni, da če za vozlišči $i$ velja 
$d(u, i) \neq d(v, i)$ in je hkrati $i \in S$ bo potem vsota enaka 2, kar pa je ravno to kar iščemo.

\section{Načrt dela}

Ideja prve faze projekta je identifikacija grafov z določenimi lastnostmi glede na njihovo metrično 
dimenzijo in na napake odporno metrično dimenzijo. V najini kodi sva na začetku napisali funkciji, 
ki izračunata metrično in na napake odporno metrično dimenzijo grafa. Za vsak graf G funkciji 
preštejeta število  vozlišč in medsebojne razdalje za poljuben par, nato pa rešita ustrezen CLP 
in vrneta moč razrešljivih množic. Funkciji bova nato uporabili za iskanje grafov, ki ustrezajo 
podanim parametrom, ki so ciljna metrična dimenzija, ciljna na napake odporna metrična dimenzija 
in največje število vozlišč, do katerega preverjamo grafe.

V naslednji fazi bi za iskanje grafov uporabile metahevristični sistem. To je vrsta algoritma v 
stohastični optimizaciji, ki uporablja naključnost za iskanje optimalnih ali čim bolj optimalnih 
rešitev za zahtevne probleme. Njihov namen ni nujno, da poišče najbolj optimalno rešitev, temveč 
da poišče dovolj dobre rešitve v razumnem času, še posebej kadar je popolna rešitev težko dosegljiva
zaradi kompleknosti problema.

Naša pričakovanja so, da grafov z matrično dimenzijo, odporno na napake ni veliko in jih je tudi 
težko poiskati. Na podlagi tega pričakujemo, da bo poskusov nekoliko več kot smo sprva pričakovale.
Poleg tega domnevamo, da čim manjša je razlika med metrično dolžino in metrično dolžino, odporno na 
napake, tem lažje je poiskati grafe. 

\end{document}