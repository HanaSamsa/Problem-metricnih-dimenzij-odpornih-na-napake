\documentclass[a4paper,10pt]{article}
\usepackage[slovene]{babel}
\usepackage[utf8]{inputenc}
\usepackage[T1]{fontenc}
\usepackage{lmodern}
\usepackage{amsmath,amsfonts}
\usepackage{enumitem}
\usepackage{footmisc}
\pagestyle{empty}

\begin{document}

\section{Metrična dimenzija}
Množica $S \in V$ je v grafu $G$ razrešljiva, če za vsak par $x, y \in V(G)$ 
ostaja $s \in S,$ da velja $d(x, s) \neq d(y, s).$ Rečemo, da sta $x$ in $y$
razrešeni z vozliščem $s$. Množica $S$ je odporna na napake, če je 
$S \setminus \{v\}$ prav tako razrešljiva za vsak $v \in S.$ 


Metrična dimenzija neusmerjenega in povezanega grafa $G = (V, E)$ je najmanjša 
podmnožica nabora vozlišč $S$ iz $V$ z lastnostjo, da so vsa vozlišča v $V$ 
enolično določena z njihovimi razdaljami do vozlišč podmnožice $S$.

Primer uporabe metrične dimenzije je problem robotske navigacije. Pri tem pusitmo
robota, da navigira v nekem prostoru, ki je določen z grafom $G$. Pri tem so 
povezave grafa $G$ poti. Robot pošlje signal do posameznega niza vozlišč imenovanih 
orientacijske točke, da ugotovi kako daleč od njih se nahaja. Pri tem je določanje 
najmanjše množice orientacijskih točk in njihov položaj, da robot lahko enolično 
določi, kje se nahaja, simetričen problemu metrične dolžine. Problem nastane, 
če ena od teh točk ne deluje pravilno, kar pomeni da robot nima dovolj informacij 
za enolično določanje svoje lokacije. Pri teh težavah nam prav pridejo metrične 
dolžine, odporne na napake. 
Nabor za razreševanje, odporen na napake zagotavlja, da tudi če ena od imenovanih 
točk ne deluje pravilnno bomo dobili prave informacije.

Metrična dimenzija $G$, odporna na napake, je velikost najmanjše razčlenujoče 
množice $S,$ odporne na napake (v primeru robotske navigacije je to nabor za 
razreševanje) in jo označimo z $ftdim(G).$


\section{Matematična formulacija}
Imamo povezan in neusmerjen graf $G = (V, E),$ kjer je $V = \{1, 2, \ldots, n\}$
množica vozlišč in $\mid E \mid = m.$ Naj bo $d(u, v)$ najkrajša pot med vozliščema
$u, v \in V.$

\end{document}
