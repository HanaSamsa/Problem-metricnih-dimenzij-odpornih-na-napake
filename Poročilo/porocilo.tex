\documentclass[12pt]{article}
\usepackage[a4paper]{geometry}
\usepackage[utf8]{inputenc}
\usepackage[T1]{fontenc}
\usepackage[slovene]{babel}
\usepackage[myheadings]{fullpage}

\usepackage{fancyhdr}
\usepackage{lastpage}
\usepackage{graphicx, wrapfig, subcaption, setspace, booktabs}
\usepackage[font=small, labelfont=bf]{caption}
\usepackage[protrusion=true, expansion=true]{microtype}
\usepackage{sectsty}
\usepackage{url, lipsum}

\usepackage{lmodern}
\usepackage{amsmath,amsfonts}
\usepackage{enumitem}
\usepackage{footmisc}


\newcommand{\HRule}[1]{\rule{\linewidth}{#1}}
\onehalfspacing
\setcounter{tocdepth}{5}
\setcounter{secnumdepth}{5}

\pagestyle{fancy}
\fancyhf{}
\setlength\headheight{15pt}

\begin{document}

\title{ \normalsize \textsc{Projektna naloga}
		\\ [2.0cm]
		\HRule{0.5pt} \\
		\LARGE \textbf{\uppercase{Problem na napake odporne metrične dimenzije}}
		\HRule{2pt} \\ [0.5cm]
		\normalsize \date{december 2024} \vspace*{5\baselineskip}}
\author{
		Anamarija Potokar, Hana Samsa 
        \vspace{1 cm} \\
		Mentorja: doc. dr. Janoš Vidali, \\
        prof. dr. Riste Škrekovski 
        \vspace{1 cm} \\
		Fakulteta za matematiko in fiziko }

\maketitle

\newpage

\section{Na napake odporna metrična dimenzija}

Množica $S \in V$ v grafu $G$ je razrešljiva, če za vsak par vozlišč $x, y \in V(G)$ 
ostaja vozlišče $s \in S,$ da velja $d(x, s) \neq d(y, s).$ Rečemo, da sta $x$ in $y$
razrešeni z vozliščem $s$. Množica $S$ je odporna na napake, če je 
$S \setminus \{v\}$ prav tako razrešljiva za vsak $v \in S.$ 

Metrična dimenzija neusmerjenega in povezanega grafa $G = (V, E)$ je najmanjša 
podmnožica nabora vozlišč $S \subset V$ z lastnostjo, da so vsa vozlišča v $V$ 
enolično določena z njihovimi razdaljami do vozlišč podmnožice $S$.

%Primer uporabe metrične dimenzije je problem robotske navigacije. Pri tem pusitmo
%robota, da navigira v nekem prostoru, ki je določen z grafom $G$. Pri tem so 
%povezave grafa $G$ poti. Robot pošlje signal do posameznega niza vozlišč imenovanih 
%orientacijske točke, da ugotovi kako daleč od njih se nahaja. Pri tem je določanje 
%najmanjše množice orientacijskih točk in njihov položaj, da robot lahko enolično 
%določi, kje se nahaja, simetričen problemu metrične dolžine. Problem nastane, 
%če ena od teh točk ne deluje pravilno, kar pomeni da robot nima dovolj informacij 
%za enolično določanje svoje lokacije. Pri teh težavah nam prav pridejo metrične 
%dolžine, odporne na napake. 
%Nabor za razreševanje, odporen na napake zagotavlja, da tudi če ena od imenovanih 
%točk ne deluje pravilnno bomo dobili prave informacije.

Na napake odporna metrična dimenzija grafa $G$, je velikost najmanjše razčlenujoče 
množice $S,$ odporne na napake in jo označimo z $ftdim(G).$
\vspace{0,5 cm}

Naloga projektne naloge je bila, da s pomočjo celoštevilskega linearnega programa 
poiščemo grafe z $dim(G) = 2$ in $ftdim(G) = 5, 6, 7$ ali več. Pri tem se za manjše
grafe, torej grafe z malo vozlišči, uporablja sistematično iskanje (ang.\textit{
systematic search}), za večje grafe pa metahevristični pristop (ang.\textit{ simulated 
annealing search}).

\section{Potek dela}


\section{Ugotovitve}


\end{document}