\documentclass[12pt]{article}
\usepackage[a4paper]{geometry}
\usepackage[utf8]{inputenc}
\usepackage[T1]{fontenc}
\usepackage[slovene]{babel}
\usepackage[myheadings]{fullpage}

\usepackage{hyperref}
\usepackage{fancyhdr}
\usepackage{lastpage}
\usepackage{graphicx, wrapfig, subcaption, setspace, booktabs}
\usepackage[font=small, labelfont=bf]{caption}
\usepackage[protrusion=true, expansion=true]{microtype}
\usepackage{sectsty}
\usepackage{url, lipsum}
\usepackage{float}

\usepackage{lmodern}
\usepackage{amsmath,amsfonts}
\usepackage{enumitem}
\usepackage{footmisc}


\newcommand{\HRule}[1]{\rule{\linewidth}{#1}}
\onehalfspacing
\setcounter{tocdepth}{5}
\setcounter{secnumdepth}{5}

\pagestyle{fancy}
\fancyhf{}
\setlength\headheight{15pt}

\begin{document}

\title{ \normalsize \textsc{Projektna naloga}
		\\ [2.0cm]
		\HRule{0.5pt} \\
		\LARGE \textbf{\uppercase{Problem na napake odporne metrične dimenzije}}
		\HRule{2pt} \\ [0.5cm]
		\normalsize \date{December 2024} \vspace*{5\baselineskip}}
\author{
		Anamarija Potokar, Hana Samsa 
        \vspace{1 cm} \\
		Mentorja: doc. dr. Janoš Vidali, \\
        prof. dr. Riste Škrekovski 
        \vspace{1 cm} \\
		Fakulteta za matematiko in fiziko }

\maketitle

\newpage

\section{Na napake odporna metrična dimenzija}

Množica $S \in V$ v grafu $G$ je razrešljiva, če za vsak par vozlišč $x, y \in V(G)$ 
ostaja vozlišče $s \in S,$ da velja $d(x, s) \neq d(y, s).$ Rečemo, da sta $x$ in $y$
razrešeni z vozliščem $s$. Množica $S$ je odporna na napake, če je 
$S \setminus \{v\}$ prav tako razrešljiva za vsak $v \in S.$ 

Metrična dimenzija neusmerjenega in povezanega grafa $G = (V, E)$ je najmanjša 
podmnožica nabora vozlišč $S \subset V$ z lastnostjo, da so vsa vozlišča v $V$ 
enolično določena z njihovimi razdaljami do vozlišč podmnožice $S$.

%Primer uporabe metrične dimenzije je problem robotske navigacije. Pri tem pusitmo
%robota, da navigira v nekem prostoru, ki je določen z grafom $G$. Pri tem so 
%povezave grafa $G$ poti. Robot pošlje signal do posameznega niza vozlišč imenovanih 
%orientacijske točke, da ugotovi kako daleč od njih se nahaja. Pri tem je določanje 
%najmanjše množice orientacijskih točk in njihov položaj, da robot lahko enolično 
%določi, kje se nahaja, simetričen problemu metrične dolžine. Problem nastane, 
%če ena od teh točk ne deluje pravilno, kar pomeni da robot nima dovolj informacij 
%za enolično določanje svoje lokacije. Pri teh težavah nam prav pridejo metrične 
%dolžine, odporne na napake. 
%Nabor za razreševanje, odporen na napake zagotavlja, da tudi če ena od imenovanih 
%točk ne deluje pravilnno bomo dobili prave informacije.

Na napake odporna metrična dimenzija grafa $G$, je velikost najmanjše razčlenujoče 
množice $S,$ odporne na napake in jo označimo z $ftdim(G).$
\vspace{0,5 cm}

Naloga projektne naloge je bila, da s pomočjo celoštevilskega linearnega programa 
poiščemo grafe z $dim(G) = 2$ in $ftdim(G) = 5, 6, 7$ ali več. Pri tem se za manjše
grafe, torej grafe z malo vozlišči, uporablja sistematično iskanje (ang.\textit{
systematic search}), za večje grafe pa metahevristični pristop (ang.\textit{ simulated 
annealing search}).

\section{Celoštevilski linearni program}

Imamo povezan in neusmerjen graf $G = (V, E),$ kjer je $V = \{1, 2, \ldots, n\}$
množica vozlišč in $\mid E \mid = m.$ Naj bo $d(u, v)$ najkrajša pot med vozliščema
$u, v \in V.$

V naslednjem celoštevilskem lineearnem programu naj velja $1 \leq u < v \leq n$ in 
$1 \leq i < j \leq n.$ Najprej definirajmo matriko koeficientov A. Spremenljivka $x_i,$ 
definirana s predpisom \eqref{eq:2} nam pove ali vozlišče $i$ pripada množici $S.$ 

\begin{equation}
    A_{(u, v), i} = \begin{cases}
        1, d(u, i) \neq d(v, i) \\
        0, d(u, i) = d(v, i)
    \end{cases}
\label{eq:1}
\end{equation}

\begin{equation}
    x_i = \begin{cases}
        1, i \in S \\
        0, i \notin S
    \end{cases}
\label{eq:2}
\end{equation} 

\begin{equation}
    \min \sum_{i = 1}^{n} x_i  
\label{eq:3}
\end{equation}

\begin{equation}
    \sum_{i = 1}^{n} A_{(u, v), i} \cdot x_i \geq 2, \text{ } 1 \leq u < v \leq n 
\label{eq:4}
\end{equation}

\begin{equation}
    x_i \in \{0, 1\}, \text{ } 1 \leq i \leq n
\label{eq:5}
\end{equation}

Enačba \eqref{eq:3} predstavlja najmanjšo podmnožico $S.$ Enačba \eqref{eq:4} pa nam da pogoj, da
je poodmnožica $S$ razrešljiva množica, odporna na napake. To pomeni, da če za vozlišči $i$ velja 
$d(u, i) \neq d(v, i)$ in je hkrati $i \in S$ bo potem vsota enaka 2, kar pa je ravno to kar iščemo.

\section{Potek dela}


\section{Koda}
Komentirana koda je dostopna na \href{https://github.com/HanaSamsa/Problem-metricnih-dimenzij-odpornih-na-napake.git}{povezavi}.

\section{Sistematično iskanje}

V prvi fazi sva se iskanja ustreznih grafov z lastnostima dim($G$)$ =2$ in ftdim($G$) = 5, 6, 7, ... 
lotili tako, da sva za konstantno vrednost dim postopoma povečevali željeno ftdim in število vozlišč, za 
katerega iščemo ustrezne grafe. Najprej sva kodo preizkusili za vrednosti dim = 2 in ftdim = 4:
V prvi fazi sva se iskanja ustreznih grafov z lastnostima $dim(G) =2$ in $ftdim(G) = 5, 6, 7, ...$ lotili tako, da sva za 
konstantno vrednost dim postopoma povečevali željeno ftdim in število vozlišč, za katerega iščemo ustrezne grafe. Najprej sva 
kodo preizkusili za vrednosti $dim = 2$ in $ftdim = 4$:
\begin{itemize}
    \item na $4$ vozliščih obstajata $2$ taka grafa, čas izvajanja kode je $0.04$ sekunde,
    \item na $5$ vozliščih obstaja $8$ takih grafov, čas izvajanja kode je $0.11$ sekunde,
    \item na $6$ vozliščih obstaja $46$ takih grafov, čas izvajanja kode je $0.69$ sekunde,
    \item na $7$ vozliščih obstaja $232$ takih grafov, čas izvajanja kode je $7.19$ sekund,
    \item na $8$ vozliščih obstaja $1525$ takih grafov, čas izvajanja kode je $2$ minuti in $1$ sekunda.
\end{itemize}

Rezultate za $ftdim = 4$ ponazorimo še s tabelo:

\begin{table}[H] 
    \centering 
 	\begin{tabular}{|c|c|c|} 
 	\hline 
 	\textbf{št. vozlišč} & \textbf{št. grafov} & \textbf{čas izvajanja} \\
 		\hline 4 & 2 & 0.04s  \\ 
 		\hline 5 & 8 & 0.11s \\ 
 		\hline 6 & 46 & 0.69s \\ 
 		\hline 7 & 232 & 7.19s \\
 		\hline 8 & 1525 & 2min1s \\
 		\hline 
 	\end{tabular} 
 	\caption{Rezultati za ftdim = 4}
 	\label{tab:ftdim4}
\end{table}


Nato sva se lotili iskanja odgovora na vprašanje naloge: za $ftdim(G) = 5$ sva najprej ugotovili, da za manj kot $7$ 
vozlišč tak graf sploh ne obstaja. Na $7$ vozliščih obstajata dva ustrezna grafa, katera je algoritem našel in izrisal v 
$5.96$ sekundah.

\begin{figure}[H]
    \centering
    \includegraphics[width=0.25\textwidth]{C:/Users/Hana/Desktop/Hana/faks/Finančni praktikum/Seminarska naloga/257_1.png}
    \caption{Graf pri parametrih $dim = 2,$ $ftdim = 5,$ in število vozlišč je $7.$}
    \label{fig:slika257_1}
\end{figure}

\begin{figure}[H]
    \centering
    \includegraphics[width=0.25\textwidth]{C:/Users/Hana/Desktop/Hana/faks/Finančni praktikum/Seminarska naloga/257_2.png}
    \caption{Graf pri parametrih $dim = 2,$ $ftdim = 5,$ in število vozlišč je $7.$}
    \label{fig:slika257_2}
\end{figure}

Na $8$ vozliščih obstaja že bistveno več grafov, za katere velja $dim(G) = 2$ in $ftdim(G) = 5$, in sicer $65$, tudi koda 
pa v primerjavi z grafi na $7$ vozliščih rabi veliko dlje časa, da se izvede; ustrezne grafe je poiskala in izrisala v $1$ 
minuti in $58$ sekundah. 

\begin{figure}[H]
    \centering
    \includegraphics[width=0.25\textwidth]{C:/Users/Hana/Desktop/Hana/faks/Finančni praktikum/Seminarska naloga/258_1.png}
    \caption{Graf pri parametrih $dim = 2,$ $ftdim = 5,$ in število vozlišč je $8.$}
    \label{fig:slika258_1}
\end{figure}

\begin{figure}[H]
    \centering
    \includegraphics[width=0.25\textwidth]{C:/Users/Hana/Desktop/Hana/faks/Finančni praktikum/Seminarska naloga/258_2.png}
    \caption{Graf pri parametrih $dim = 2,$ $ftdim = 5,$ in število vozlišč je $8.$}
    \label{fig:slika258_2}
\end{figure}

\begin{figure}[H]
    \centering
    \includegraphics[width=0.25\textwidth]{C:/Users/Hana/Desktop/Hana/faks/Finančni praktikum/Seminarska naloga/258_3.png}
    \caption{Graf pri parametrih $dim = 2,$ $ftdim = 5,$ in število vozlišč je $8.$}
    \label{fig:slika258_3}
\end{figure}

\begin{figure}[H]
    \centering
    \includegraphics[width=0.25\textwidth]{C:/Users/Hana/Desktop/Hana/faks/Finančni praktikum/Seminarska naloga/258_4.png}
    \caption{Graf pri parametrih $dim = 2,$ $ftdim = 5,$ in število vozlišč je $8.$}
    \label{fig:slika258_4}
\end{figure}

\begin{figure}[H]
    \centering
    \includegraphics[width=0.25\textwidth]{C:/Users/Hana/Desktop/Hana/faks/Finančni praktikum/Seminarska naloga/258_5.png}
    \caption{Graf pri parametrih $dim = 2,$ $ftdim = 5,$ in število vozlišč je $8.$}
    \label{fig:slika258_5}
\end{figure}

\begin{figure}[H]
    \centering
    \includegraphics[width=0.25\textwidth]{C:/Users/Hana/Desktop/Hana/faks/Finančni praktikum/Seminarska naloga/258_6.png}
    \caption{Graf pri parametrih $dim = 2,$ $ftdim = 5,$ in število vozlišč je $8.$}
    \label{fig:slika258_6}
\end{figure}

\begin{figure}[H]
    \centering
    \includegraphics[width=0.25\textwidth]{C:/Users/Hana/Desktop/Hana/faks/Finančni praktikum/Seminarska naloga/258_7.png}
    \caption{Graf pri parametrih $dim = 2,$ $ftdim = 5,$ in število vozlišč je $8.$}
    \label{fig:slika258_7}
\end{figure}

Pri iskanju in risanju takih grafov na 9 vozliščih se je koda po pol ure dela prenehala izvajati in ni našla vseh ustreznih 
grafov. 

\begin{figure}[H]
    \centering
    \includegraphics[width=0.25\textwidth]{C:/Users/Hana/Desktop/Hana/faks/Finančni praktikum/Seminarska naloga/259_1.png}
    \caption{Graf pri parametrih $dim = 2,$ $ftdim = 5,$ in število vozlišč je $9.$}
    \label{fig:slika259_1}
\end{figure}

\begin{figure}[H]
    \centering
    \includegraphics[width=0.25\textwidth]{C:/Users/Hana/Desktop/Hana/faks/Finančni praktikum/Seminarska naloga/259_2.png}
    \caption{Graf pri parametrih $dim = 2,$ $ftdim = 5,$ in število vozlišč je $9.$}
    \label{fig:slika259_2}
\end{figure}

\begin{figure}[H]
    \centering
    \includegraphics[width=0.25\textwidth]{C:/Users/Hana/Desktop/Hana/faks/Finančni praktikum/Seminarska naloga/259_3.png}
    \caption{Graf pri parametrih $dim = 2,$ $ftdim = 5,$ in število vozlišč je $9.$}
    \label{fig:slika259_3}
\end{figure}

Rezultate za $ftdim = 5$ ponazorimo še s tabelo:

\begin{table}[H]
	\centering 
 	\begin{tabular}{|c|c|c|} 
 		\hline 
 		\textbf{št. vozlišč} & \textbf{št. grafov} & \textbf{čas izvajanja} \\
 		\hline 7 & 2 & 5.96s  \\ 
 		\hline 8 & 65 & 1min58s \\
 		\hline 
 	\end{tabular} 
 	\caption{Rezultati za $ftdim = 5$}
 	\label{tab:osebe}
\end{table}



Očitno se torej že pri $ftdim = 5$ in $9$ vozliščih, kar se zdi dokaj malo, koda ne izvede v sprejemljivem času. Preverili 
sva še, kako časovno zahtevno je iskanje grafov z $dim = 2$ in $ftdim = 6$, za kar je potrebnih najmanj $12$ vozlišč; tudi 
ta koda se ni izvedla v normalnem času, zato sva za primere pri $ftdim = 5$ in več kot $8$ vozlišč ter $ftdim = 6, 7, ...$ 
grafe iskali s pomočjo metahevrističnih algoritmov. 


\section{Metahevristično iskanje}

V naslednji fazi sva želeli narediti kodo, ki bi poiskala tudi večje grafe. To sva dosegli s metahevrističnim pristopom. Tako
kot v prvi fazi sva kodo najprej preizkusili za $dim = 2$ in $ftdim = 4.$ Pri tem sva za grafe vzeli že večje grafe s številom
vozlišč od $9$ do $15,$ poskusili pa sva tudi za grafe z $20$ vozlišči.
 
Pri metahevrističnem iskanju se je pojavila težava pri kodah metricna\_dolzina in na\_napake\_odporna\_metricna\_dolzina, zato 
je ta koda drugačna kot je pri sistematičnem iskanju. 






\end{document}
